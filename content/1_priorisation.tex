\chapter{Priorisierung}\label{ch:priorisation}

\section{Drone Technology in Architecture, Engineering, and Construction: A Strategic Guide to Unmanned Aerial Vehicle Operation and Implementation: Chapter 2 - A Paradigm Shift in Viewing the World}

Dieses Buch erläutert und diskutiert eine große Bandbreite von möglichen Anwendungen unbemannter Flugzeuge, sogenannter \textit{unmanned aerial vehicles} (UAV) und wie sie in den letzten Jahrzehnten Einzug in verschiedene Industriezweige gefunden haben.
Besonders Kapitel zwei, \textit{A Paradigm Shift in Viewing the World}, bietet einen strukturierten Überblick über die Entwicklung der Drohnentechnologie, die aktuellen und zukünftigen Anwendungsfälle in der Industrie sowie die Risiken, die die Anwendung von Drohnen bei untrainierter Nutzung birgt \cite{Tal2021}.

\section{TRENDS, BENEFITS, AND BARRIERS OF UNMANNED AERIAL SYSTEMS IN THE CONSTRUCTION INDUSTRY: A SURVEY STUDY IN THE UNITED STATES}

Weitere tiefgreifendere Informationen zur Analyse von Drohnentechnologien zur Digitalisierung von Prozessen in der Bauwirtschaft mit Fokus auf die Zukunftschancen in Anbetracht von Kosten und Nutzen, bietet die Veröffentlichung \textit{Trends, benefits, and barriers of unmanned aerial systems in the construction industry: A survey study in the United States} in dem \textit{Journal of Information Technology in Construction}.
Diese Umfrage-Studie beleuchtet die häufigsten Anwendungsgebiete von unbemannten Flugobjekten in der amerikanischen Industrie sowie die Vorteile und Einschränkungen, die sich Unternehmen in den vereinigten Staaten bei der Integration von Drohnen in ihre bestehenden Arbeitsprozesse ausgesetzt sehen \cite{abaeano2021trends}.