\chapter{Zusammenfassung Teil 2}\label{ch:summary2}

\section{Introduction}

Ähnlich wie Tal leiten Albeaino und Gheisari ihre Arbeit mit der historischen Entwicklung von unbemannten Flugobjekten ein.
Nach langer militärischer Nutzung solcher unbemannter Flugobjekte, so erklären sie, gab es es signifikante Fortschritte in der Hard- und Software Entwicklung von Drohnen, wodurch sie seit Ende des 20. Jahrhunderts zunehmend Einzug in die Privatwirtschaft erfahren.
Vorreiterbranchen neben dem Bauwesen sind unter anderem Architektur, Infrastruktur Management, Verkehrsüberwachung, Landwirtschaft, Logistik, Notfall-/Rettungswesen und Sicherheitsdienste.
Spezielle Arbeitsprozesse, die im Bauwesen mittels Drohnen digitalisiert und beschleunigt werden können, sind nach den Autoren die Geländevermessung, Baufortschrittsüberwachung und Inspektion von Bauwerken.
Diese Prozesse sind besonders geeignet für den Einsatz von Drohnen, da diese den sicheren Zugang zu schwer erreichbaren oder gefährlichen Standorten ermöglichen ohne den Piloten in Gefahr zu bringen. \cite[S. 84--85]{abaeano2021trends}

% \begin{itemize}
%     \item Drohnen ursprünglich nur militärische Anwendung
%     \item signifikante Entwicklungen bei HW und SW
%     \item Einzug in die Privatwirtschaft
%     \item Vorreiterbranchen sind AEC, Infrastruktur Management (inspection, planning), Verkehrsüberwachung, Landwirtschaft (Ackerlandanalyse, Pflanzengesundheit Monitoring), Materiallierferung, Such-/Bergungsaktionen, Überwachung
%     \item nach einer Studie 2017, die zitiert wird, soll der Markt für unbemannte Flugkörper einen Jährlichen Umsatz von 30 Milliarden Dollar im Jahr 2026 erreichen.
% \end{itemize}
% \vspace{-\baselineskip}

% Drohnen (UAS) tragen erheblich zur Digitalisierung im Bauwesen bei, indem sie vielfältige Anwendungen in Architektur, Ingenieur- und Bauprojekten ermöglichen.
% In Bereichen wie der Verkehrsüberwachung, Hangüberwachung, dem Erhalt von Kulturerbe und der Stadtplanung kommen Drohnen zunehmend zum Einsatz.
% Besonders in der Bauindustrie bieten sie Vorteile durch sicheren Zugang zu schwer erreichbaren oder gefährlichen Standorten und erlauben eine effiziente und kostengünstige Projektumsetzung.
% Fortschritte in der Regulierung, Sensorik und autonomen Steuerung fördern die breite Anwendung von Drohnen besonders in der Geländevermessung, Baufortschrittsüberwachung und Inspektion von Bauwerken (vgl. Albeaino et al., 2019; Gheisari und Esmaeili, 2019).

% Es wird erklärt, dass verschiedene Umfragen und Beobachtungen die Verbreitung von Drohnen im Bauwesen untersuchen, jedoch nie eine vollständige Einordnung in die Entwicklung, Hindernisse und Vorteile der Integration von unbemannten Flugobjekten in die Bauindustrie durchgeführt wird.
% Das Ziel des Papers ist es dies durchzuführen. \cite[S. 85--86]{abaeano2021trends}

\section{Background}
\subsection{UAS Application Areas}
Im Anschluss stellen sich die Autoren der Fragestellung, in welchen Arbeitsschritten eines Projekts aus dem Bauwesen Drohnen Anwendung finden, um zu zeigen, dass Drohnen keine Nischenerscheinung auf der Baustelle sind.
Denn den Ausführungen von Albeaino und Gheisari zufolge, bringen Drohnen bereits vor Beginn der physischen Arbeiten auf der Baustelle Vorteile mit sich.
Während und Nach den Arbeiten ebenfalls. \cite[S. 86]{abaeano2021trends}

Durchwegs können mittels Drohnen, die mit diversen Erweiterungen wie Sensoren und Kameras ausgestattet sind, traditionelle Vorgehen zur Vermessung, Inspektion und Bauüberwachung durch effizientere oder kostengünstigere Vorgehen ersetzt werden.
Als maßgebliche Faktoren dafür haben Albeaino und Gheisari die Flexibilität von Drohnen, die Möglichkeit mittels Drohnen sehr hochauflösende Bilder zu erstellen und die hohe Arbeitsnähe von Drohnen zum Beispiel bei der nicht destruktiven Inspektion von Gebäudeteilen identifiziert.
Um diese Schlussfolgerung zu bestätigen zitieren sie einige Experimente aus verschiedenen Veröffentlichungen anderer Wissenschaftler. \cite[S. 87--88]{abaeano2021trends}

Die Anwendungsfälle nach einer Baustelle beziehen hauptsächlich auf die Bereiche der Instandhaltung und Schadensbewertung. \cite[S. 89]{abaeano2021trends}

% Unbemannte Flugsysteme finde Anwendung in allen Bereichen einer Baustelle.
% Ob vor Beginn, während oder nach der der Baustelle.
% VOR Traditionelle Methoden für die Vermessung wie Laserscanner und LiDAR sin teurer und zeitaufwändiger.

% WÄHREND:
% Drohnen, oder unbemannte Luftfahrtsysteme (UAS), bringen durch vielfältige Anwendungen bedeutende Fortschritte im Bauwesen. Sie optimieren Bauinspektionen, Fortschrittsüberwachung und Sicherheitsüberwachung und sparen dabei Kosten, Zeit und Risiken gegenüber traditionellen Methoden. Beispielsweise nutzten Eschmann et al. (2012) Drohnen zur Rissdetektion an Fassaden mit millimetergenauer Präzision, während Unger et al. (2014) den Baufortschritt von Gebäuden über mehrere Monate dokumentierten. Im Sicherheitsbereich entwickelten Roberts et al. (2017) ein Drohnensystem, das potenzielle Gefahren bei Kranarbeiten erkennt. Durch solche Innovationen erhöhen Drohnen die Effizienz und Sicherheit auf Baustellen erheblich.

% Flexibilität, hoch auflösende Bilder, Arbeitsnähe ermöglicht Monitoring

% NACH:
% Bei der Schadensbewertung ermöglichen sie, auch unter gefährlichen Bedingungen sicher detaillierte zweidimensionale und dreidimensionale Karten zu erstellen und strukturelle Schäden präzise zu dokumentieren, wie z.B. bei Erdbeben in Italien und Japan (Kruijff et al., 2012; Michael et al., 2012).
% Moderne Verfahren wie die automatische Erkennung von Schäden anhand thermaler Schrägbilder erreichen eine Genauigkeit von rund 80 \% (Zhang et al., 2020).
% In der Gebäudewartung erleichtern UAS die Inspektion schwer zugänglicher Fassaden und Dächer, reduzieren dabei Zeit- und Kostenaufwand und bieten durch flexible Datenerfassung Unterstützung für die Instandhaltungsplanung und energetische Analysen (Liu et al., 2016; Chen et al., 2021).

% Site Mapping/Surveying, Site Planning, Building Inspection, Progress Monitoring, Safety Management, Earthmoving, Aerial Construction, Material Handling, Security Surveillance, Ste Communication, Post-Disaster reconnaissance, Building Maintenance

\subsection{UAS Technology}
% Die beiden am häufigsten eingesetzten Drohnentypen sind Drehflügler (z.B. Multicopter) und Starrflügler. Drehflügler sind wendig und eignen sich besonders für kleinere Bauprojekte und vertikale Strukturen, während Starrflügler eine hohe Reichweite bieten und große horizontale Bauprojekte effizient überwachen können.

% Der Einsatz von Navigationsmodi wie autonom, semi-autonom und manuell sowie von Software zur Steuerung und Datenauswertung optimiert die Effizienz und Sicherheit der Flugmissionen. Autonome Funktionen wie automatische Rückkehr und Hindernisvermeidung erhöhen die Betriebssicherheit und erleichtern die Überwachung von Baustellen. Unterschiedliche Sensoren, darunter Wärmebildkameras, LiDAR und RFID, erweitern die Anwendungsbereiche durch präzise Datenerfassung, z.B. für Bauwerksüberwachung, Schadenserfassung und Materialnachverfolgung.

Ein weiterer Aspekt ist die Verbreitung verschiedener Drohnentypen im Bauwesen.
Denn verschiedene Situationen stellen unterschiedliche Anforderungen an eine Drohne.
So werden zwar überwiegend Multicopter mit vier bis sechs Rotoren verwendet, jedoch sind für das Überfliegen sehr großer Gebiete Drohnen mit starren Flügeln geeigneter.
Die häufigst genutzten Erweiterungen sind dabei RGB- und Wärmebild-Kameras.
\cite[S. 90--92]{abaeano2021trends}


\subsection{UAS Technology UAS Regulation and Training Requirements}
Bezüglich der Regulierung des Luftraums und der Nutzung von Drohnen sind in den vereinigten Staaten von Amerika die \textit{United States Federal Aviation Administration} (FAA) zuständig.
Albeaino und Gheisari beschreiben kurz die zur Umfrage aktuellen Regulierungen zum Flug mit Drohnen in den vereinigten Staaten.
Dabei ergibt sich, dass es schon einige Vorgaben gibt, diese aber zu streng sind und somit den Einsatz von Drohnen in der Industrie behindern.
Auf die Regeln für den Flug mit Drohnen in anderen Staaten gehen sie nicht ein. \cite[S. 92]{abaeano2021trends}

\section{Research Methods}
Im Kapitel für die Recherchemethoden erklären die Autoren die Vorbereitung, den Ablauf und die Zielgruppe für ihre Umfrage. \cite[S. 93]{abaeano2021trends}

\section{Results}
\subsection{Application Trends of UAS Integration in Construction}
% marketing, progress monitoring, Site planning, site mapping and surveying
% vs
% security surveillance, material handling, site Communication

% It is also worth noting that FAA rules prohibit UAS operations during nighttime and
% out of visual line-of-sight which might currently limit its deployment as a security surveillance tool on construction jobsites.
% However, given the rate of technological advancements and regulation updates, it is more likely that these types of applications overcome the presented limitations and become more frequently implemented in the near future.

Die Auswertung der Umfrage gibt folgend tiefe Einblicke in die Verbreitung von unbemannten Flugsystemen in der amerikanischen Industrie.
Den Ergebnissen von Albeaino und Gheisari zufolge werden Drohnen am häufigsten für Fortschrittsüberwachung (84\% der Befragten), Baustellenplanung (68\% der Befragten) und zur Kartierung/Vermessung (61\% der Befragten) verwendet.
Für diese Ziele benutzen die Befragten Drohnen mehrheitlich zur Aufnahme von Fotos (93\%), folgend von Videos (84\%) und zwar mittels regulärer Kameras (93\%) und anschließend Wärmebildkameras (34\%). \cite[S. 95--97]{abaeano2021trends}

\subsection{Benefits of UAS Integration in Construction}
% Die Integration von Unmanned Aerial Systems (UAS) im Bauwesen zeigt signifikante Vorteile in Bezug auf Zeit- und Kostenersparnis sowie verbesserten Zugang zu schwer erreichbaren Bereichen. Die Umfrageergebnisse weisen auf eine starke Zustimmung der Fachleute hin, dass UAS sowohl die Effizienz (Zeit: 4,39 ± 0,94; Kosten: 4,27 ± 0,97) als auch die Zugänglichkeit (4,38 ± 0,96) verbessern. Zudem wird eine moderate Verbesserung der Arbeitsqualität (3,93 ± 1,00) und eine Reduzierung von Sicherheitsbedenken (3,88 ± 1,12) festgestellt.

Weiterhin haben nach dem Fragenkatalog der Autoren die Befragten angegeben, dass von den Gründen für den Einsatz von Drohnen, die Zeitersparnis, die Kostenersparnis und die verbesserte Zugänglichkeit der Baustelle am schwerwiegendsten sind. \cite[S. 98--99]{abaeano2021trends}

\section{Barriers}
% Problems with weather, Herausforderungen im Zusammenhang mit beengten oder überfüllten Bereichen, know-how, liability and legal challenges, training and certification, night use

% Awareness und Verfügbarkeit von Trainings speziell für Drohnen im Bauwesen

Zu den Vorteilen präsentieren Albeaino und Gheisari auch mehrere Hindernisse, die den Einsatz von Drohnen in der Industrie behindern.
Zu diesen Hindernissen gehören den Antworten zufolge, dass Drohnen sehr anfällig gegenüber dem Wetter sind, dass der Einsatz von Drohnen Know-How benötigt, das nicht vorhanden ist, dass der Nutzen von Drohnen in beengten Bereichen eingeschränkt ist und dass der Einsatz von Drohnen Probleme bezüglich der Haftbarkeit birgt.
Weiterhin fehlen auch Trainingsressourcen für die Anwendung und die Piloten von Drohnen speziell im Bauwesen.
Weiterhin können die Ergebnisse der Umfrage verbessert werden, indem zum Beispiel die Anzahl der Befragten erhöht wird.
Eine weiter Option ist die Umfrage in Zukunft wiederholt durchzuführen, um die Trends besser abzubilden. \cite[S. 99--104]{abaeano2021trends}

\section{Conclusion}

% Weitere Entwicklungen werden UAVs noch zugänglicher machen und die bereits bestehenden Vorteile nur verstärken.

Anhand der Informationen, die Albeaino und Gheisari in der Umfrage gesammelt haben, schließen sie darauf, dass sich Drohnen im Bauwesen weiter ausbreiten werden.
Ein treibender Faktor ist dabei die anhaltende Innovation von Hard- und Software der Drohnen.
In Zukunft werden sich auch Gesetzgeber auf Anforderungen aus der Industrie bezüglich unbemannter Flugsysteme anpassen, was die Verbreitung weiter beschleunigen wird. \cite[S. 104--105]{abaeano2021trends}


% \section{Research Limitations}