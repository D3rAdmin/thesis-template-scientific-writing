\chapter{Zusammenfassung Priorität 2}\label{ch:summary1}

\section{A Paradigm Shift in Viewing the World}
In \textit{A Paradigm Shift in Viewing the World} beschreibt Tal die historische Entwicklung von unbemannten Flugobjekten (UAV), beginnend im 19. Jahrhundert mit Ballons bis hin zu modernen UAVs beziehungsweise Drohnen.
Tal argumentiert, dass die einfache Bedienbarkeit und die mit Innovation verbundenen sinkenden Kosten der Drohnen zur professionellen Nutzung in Bereichen wie Landwirtschaft und Überwachung geführt haben \cite[p. 11]{Tal2021}.

\subsection{The Breath of Drone Applications}
\paragraph{Basic Drone Use}
In \textit{The Breath of Drone Applications} untersucht Tal die breite Anwendung von Drohnen \cite[p. 12]{Tal2021}.
In dem Unterabschnitt \textit{Basic Drone Use} wird hervorgehoben, dass Drohnen Luftaufnahmen für ein breiteres Publikum zugänglich machen, was zu einem Wandel in der Präsentation von Bauprojekten und dem Design dieser führt.
Designer müssen ihre Arbeitsweise anpassen, um den neuen Anforderungen gerecht zu werden \cite[p. 12]{Tal2021}.

\paragraph{Current Breath of Drone Use}
Unter \textit{Current Breath of Drone Use} stellt Tal aktuelle Drohneneinsatzmöglichkeiten dar, wie die Nutzung von Multikoptern mit optischen und Radarsensoren.
Diese ermöglichen das Sammeln spezifischer Daten, die der Mensch sonst nur mit hohem Risiko, mit hohen Kosten oder nicht erfassen könnte.
Tal betont, dass Drohnen zunehmend in Nischenmärkten Anwendung finden.
Neue Entwicklungen wie Lieferdrohnen und fliegende Taxen werden ebenfalls thematisiert \cite[pp. 13-15]{Tal2021}.

\paragraph{Future Breath of Drone Use}
In \textit{Future Breath of Drone Use} prognostiziert Tal eine Zukunft mit autonomen Drohnen, die Daten automatisiert erfassen und verarbeiten.
Dies könnte in Bereichen wie Baustellenüberwachung und Naturschutz Anwendung finden.
Der größte Hemmfaktor für diese Entwicklung sei jedoch laut Tal eine zu strikte Regulierung \cite[p. 16]{Tal2021}.

\subsection{The Risk of Drone Technology}
Unter \textit{The Risk of Drone Technology} werden Sicherheitsrisiken, wie der Erhalt des Luftraums und Datenschutz, thematisiert.
Tal warnt vor ungeschulten Nutzern, die schlechte Daten generieren könnten \cite[pp. 17-18]{Tal2021} \cite[p. 20]{Tal2021}.

\subsection{Why use Drones?}
In \textit{Why use Drones?} diskutiert Tal die vielfältigen Anwendungsbereiche von Drohnen und die niedrigen Kosten, die den Einsatz attraktiv machen. Unternehmen müssen jedoch abwägen, ob die Integration von Drohnen sinnvoll ist \cite[p. 21]{Tal2021}.

\subsection{The Bottom Line on Drones?}
In \textit{The Bottom Line on Drones} erklärt Tal, dass die Bauindustrie inzwischen stark von Drohnendaten abhängig ist.
Bei vielen Dienstleistungen in diesem Sektor ist es bereits Standard Drohnen für bestimmte Arbeitsschritte Drohnen einzusetzen.
Unternehmen, die solche Dienstleistungen anbieten, müssen daher entweder selbst Drohnen einsetzen oder externe Anbieter beauftragen \cite[p. 22]{Tal2021}.

% \section{Trends, benefits, and barriers of unmanned aerial systems in the construction industry: a survey study in the {United States}}
