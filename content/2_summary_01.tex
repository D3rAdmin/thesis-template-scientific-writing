\chapter{Zusammenfassung Priorität 2}\label{ch:summary1}

\section{Ein Paradigmenwechsel in der Betrachtung der Welt}
In \textit{Ein Paradigmenwechsel in der Betrachtung der Welt} beschreibt Tal die historische Entwicklung von unbemannten Flugobjekten (UAV), beginnend im 19. Jahrhundert mit Ballons bis hin zu modernen UAVs beziehungsweise Drohnen.
Tal argumentiert, dass die einfache Bedienbarkeit und die mit Innovation verbundenen sinkenden Kosten der Drohnen zur professionellen Nutzung in Bereichen wie Landwirtschaft und Überwachung geführt haben \cite[S. 11]{Tal2021}.

\section{Die Bandbreite der Drohnenanwendungen}
\paragraph{Grundlegende Drohnennutzung}
In \textit{Die Bandbreite der Drohnenanwendungen} untersucht Tal die breite Anwendung von Drohnen \cite[S. 12]{Tal2021}.
In dem Unterabschnitt \textit{Grundlegende Verwendung von Drohnen} wird hervorgehoben, dass Drohnen Luftaufnahmen für ein breiteres Publikum zugänglich machen, was zu einem Wandel in der Präsentation von Bauprojekten und dem Design dieser führt.
Designer müssen ihre Arbeitsweise anpassen, um den neuen Anforderungen gerecht zu werden \cite[S. 12]{Tal2021}.

\paragraph{Derzeitiger Umfang der Drohnennutzung}
Unter \textit{Derzeitiger Umfang der Drohnennutzung} stellt Tal aktuelle Einsatzmöglichkeiten von Drohnen dar, wie die Nutzung von Multikoptern mit optischen und Radarsensoren.
Diese ermöglichen das Sammeln spezifischer Daten, die der Mensch sonst nur mit hohem Risiko, mit hohen Kosten oder nicht erfassen könnte.
Tal betont, dass Drohnen zunehmend in Nischenmärkten Anwendung finden.
Neue Entwicklungen wie Lieferdrohnen und fliegende Taxen werden ebenfalls thematisiert \cite[S. 13-15]{Tal2021}.

\paragraph{Zukünftiger Umfang der Drohnennutzung}
In \textit{Zukünftiger Umfang der Drohnennutzung} prognostiziert Tal eine Zukunft mit autonomen Drohnen, die Daten automatisiert erfassen und verarbeiten.
Dies könnte in Bereichen wie Baustellenüberwachung und Naturschutz Anwendung finden.
Der größte Hemmfaktor für diese Entwicklung sei jedoch laut Tal eine zu strikte Regulierung \cite[S. 16]{Tal2021}.

\section{Das Risiko der Drohnentechnologie}
Unter \textit{Das Risiko der Drohnentechnologie} werden Sicherheitsrisiken, wie der Erhalt des Luftraums und Datenschutz, thematisiert.
Tal warnt vor ungeschulten Nutzern, die schlechte Daten generieren könnten \cite[S. 17-20]{Tal2021}.

\section{Warum Drohnen einsetzen?}
In \textit{Warum Drohnen einsetzen?} diskutiert Tal die vielfältigen Anwendungsbereiche von Drohnen und die niedrigen Kosten, die den Einsatz attraktiv machen. Unternehmen müssen jedoch abwägen, ob die Integration von Drohnen sinnvoll ist \cite[S. 21]{Tal2021}.

\section{Die Quintessenz über Drohnen}
In \textit{Die Quintessenz über Drohnen} erklärt Tal, dass die Bauindustrie inzwischen stark von Drohnendaten abhängig ist.
Bei vielen Dienstleistungen in diesem Sektor ist es bereits Standard Drohnen für bestimmte Arbeitsschritte Drohnen einzusetzen.
Unternehmen, die solche Dienstleistungen anbieten, müssen daher entweder selbst Drohnen einsetzen oder externe Anbieter beauftragen \cite[S. 22]{Tal2021}.

% \section{Trends, benefits, and barriers of unmanned aerial systems in the construction industry: a survey study in the {United States}}
